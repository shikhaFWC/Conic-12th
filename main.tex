%% Run LaTeX on this file several times to get Table of Contents,
%% cross-references, and citations.

\documentclass[11pt]{book}
\usepackage{gvv}
\usepackage{tfrupee}
\usepackage{gvv-book-bkup}
%\usepackage{Wiley-AuthoringTemplate}
\usepackage[sectionbib,authoryear]{natbib}% for name-date citation comment the below line
%\usepackage[sectionbib,numbers]{natbib}% for numbered citation comment the above line

%%********************************************************************%%
%%       How many levels of section head would you like numbered?     %%
%% 0= no section numbers, 1= section, 2= subsection, 3= subsubsection %%
\setcounter{secnumdepth}{3}
%%********************************************************************%%
%%**********************************************************************%%
%%     How many levels of section head would you like to appear in the  %%
%%				Table of Contents?			%%
%% 0= chapter, 1= section, 2= subsection, 3= subsubsection titles.	%%
\setcounter{tocdepth}{2}
%%**********************************************************************%%

%\includeonly{ch01}
\makeindex

\begin{document}

\frontmatter
%%%%%%%%%%%%%%%%%%%%%%%%%%%%%%%%%%%%%%%%%%%%%%%%%%%%%%%%%%%%%%%%
%% Title Pages
%% Wiley will provide title and copyright page, but you can make
%% your own titlepages if you'd like anyway
%% Setting up title pages, type in the appropriate names here:

\booktitle{CBSE Math}

\subtitle{Made Simple}

\AuAff{G. V. V. Sharma}


%% \\ will start a new line.
%% You may add \affil{} for affiliation, ie,
%\authors{Robert M. Groves\\
%\affil{Universitat de les Illes Balears}
%Floyd J. Fowler, Jr.\\
%\affil{University of New Mexico}
%}

%% Print Half Title and Title Page:
%\halftitlepage
\titlepage

%%%%%%%%%%%%%%%%%%%%%%%%%%%%%%%%%%%%%%%%%%%%%%%%%%%%%%%%%%%%%%%%
%% Copyright Page

\begin{copyrightpage}{2023}
%Title, etc
\end{copyrightpage}

% Note, you must use \ to start indented lines, ie,
% 
% \begin{copyrightpage}{2004}
% Survey Methodology / Robert M. Groves . . . [et al.].
% \       p. cm.---(Wiley series in survey methodology)
% \    ``Wiley-Interscience."
% \    Includes bibliographical references and index.
% \    ISBN 0-471-48348-6 (pbk.)
% \    1. Surveys---Methodology.  2. Social 
% \  sciences---Research---Statistical methods.  I. Groves, Robert M.  II. %
% Series.\\

% HA31.2.S873 2004
% 001.4'33---dc22                                             2004044064
% \end{copyrightpage}

%%%%%%%%%%%%%%%%%%%%%%%%%%%%%%%%%%%%%%%%%%%%%%%%%%%%%%%%%%%%%%%%
%% Only Dedication (optional) 

%\dedication{To my parents}

\tableofcontents

%\listoffigures %optional
%\listoftables  %optional

%% or Contributor Page for edited books
%% before \tableofcontents

%%%%%%%%%%%%%%%%%%%%%%%%%%%%%%%%%%%%%%%%%%%%%%%%%%%%%%%%%%%%%%%%
%  Contributors Page for Edited Book
%%%%%%%%%%%%%%%%%%%%%%%%%%%%%%%%%%%%%%%%%%%%%%%%%%%%%%%%%%%%%%%%

% If your book has chapters written by different authors,
% you'll need a Contributors page.

% Use \begin{contributors}...\end{contributors} and
% then enter each author with the \name{} command, followed
% by the affiliation information.

% \begin{contributors}
% \name{Masayki Abe,} Fujitsu Laboratories Ltd., Fujitsu Limited, Atsugi, Japan
%
% \name{L. A. Akers,} Center for Solid State Electronics Research, Arizona State University, Tempe, Arizona
%
% \name{G. H. Bernstein,} Department of Electrical and Computer Engineering, University of Notre Dame, Notre Dame, South Bend, Indiana; formerly of
% Center for Solid State Electronics Research, Arizona
% State University, Tempe, Arizona 
% \end{contributors}

%%%%%%%%%%%%%%%%%%%%%%%%%%%%%%%%%%%%%%%%%%%%%%%%%%%%%%%%%%%%%%%%
% Optional Foreword:

%\begin{foreword}
%\lipsum[1-2]
%\end{foreword}

%%%%%%%%%%%%%%%%%%%%%%%%%%%%%%%%%%%%%%%%%%%%%%%%%%%%%%%%%%%%%%%%
% Optional Preface:

%\begin{preface}
%\lipsum[1-1]
%\prefaceauthor{}
%\where{place\\
% date}
%\end{preface}

% ie,
% \begin{preface}
% This is an example preface.
% \prefaceauthor{R. K. Watts}
% \where{Durham, North Carolina\\
% September, 2004}

%%%%%%%%%%%%%%%%%%%%%%%%%%%%%%%%%%%%%%%%%%%%%%%%%%%%%%%%%%%%%%%%
% Optional Acknowledgments:

%\acknowledgments
%\lipsum[1-2]
%\authorinitials{I. R. S.}  

%%%%%%%%%%%%%%%%%%%%%%%%%%%%%%%%
%% Glossary Type of Environment:

% \begin{glossary}
% \term{<term>}{<description>}
% \end{glossary}

%%%%%%%%%%%%%%%%%%%%%%%%%%%%%%%%
%\begin{acronyms}
%\acro{ASTA}{Arrivals See Time Averages}
%\acro{BHCA}{Busy Hour Call Attempts}
%\acro{BR}{Bandwidth Reservation}
%\acro{b.u.}{bandwidth unit(s)}
%\acro{CAC}{Call / Connection Admission Control}
%\acro{CBP}{Call Blocking Probability(-ies)}
%\acro{CCS}{Centum Call Seconds}
%\acro{CDTM}{Connection Dependent Threshold Model}
%\acro{CS}{Complete Sharing}
%\acro{DiffServ}{Differentiated Services}
%\acro{EMLM}{Erlang Multirate Loss Model}
%\acro{erl}{The Erlang unit of traffic-load}
%\acro{FIFO}{First in - First out}
%\acro{GB}{Global balance}
%\acro{GoS}{Grade of Service}
%\acro{ICT}{Information and Communication Technology}
%\acro{IntServ}{Integrated Services}
%\acro{IP}{Internet Protocol}
%\acro{ITU-T}{International Telecommunication Unit -- Standardization sector}
%\acro{LB}{Local balance}
%\acro{LHS}{Left hand side}
%\acro{LIFO}{Last in - First out}
%\acro{MMPP}{Markov Modulated Poisson Process}
%\acro{MPLS}{Multiple Protocol Labeling Switching}
%\acro{MRM}{Multi-Retry Model}
%\acro{MTM}{Multi-Threshold Model}
%\acro{PASTA}{Poisson Arrivals See Time Averages}
%\acro{PDF}{Probability Distribution Function}
%\acro{pdf}{probability density function}
%\acro{PFS}{Product Form Solution}
%\acro{QoS}{Quality of Service}
%\acro{r.v.}{random variable(s)}
%\acro{RED}{random early detection}
%\acro{RHS}{Right hand side}
%\acro{RLA}{Reduced Load Approximation}
%\acro{SIRO}{service in random order}
%\acro{SRM}{Single-Retry Model}
%\acro{STM}{Single-Threshold Model}
%\acro{TCP}{Transport Control Protocol}
%\acro{TH}{Threshold(s)}
%\acro{UDP}{User Datagram Protocol}
%\end{acronyms}

\setcounter{page}{1}

\begin{introduction}
This book links high school coordinate geometry to linear algebra and matrix analysis through solved problems.
\end{introduction}

\mainmatter
\chapter{Intersection of Conics}
\section{Chords}
\begin{enumerate}
\item Using integration, find the area of the region enclosed by the curve $ y=x^2 $, the x-axis and the ordinates $x=-2$ \text { and } $x=1$.

\item Using integration, find the area of the region enclosed by line $y=\sqrt{3}x$ semi-circle $y=\sqrt{4-x^2}$ and x-axis in first quadrant.

\item Using integration, find the area of the smaller region enclosed by the curve ${4x^2 + 4y^2} = 9$ and the line $2x + 2y =3$.

\item If the area of the regin bounded by the curve $y^2 = 4ax$ and the line $x = 4a$ is $\dfrac{256}{3}$\hspace{0.2cm} sq. units, then using integration, find the value of a, where $a>0$.

\item Find the area of the region enclosed by the curves $y^2=x$, $x=\dfrac{1}{4}$,  $y=0$ and $x=1$, using integration.

\item If the area of the region bounded bythe line $y=mx$ and the curve $x^2=y$ is $\dfrac{32}{3}$\hspace{0.2cm}sq. units, then find the positive value of m, using integration.

\item If the area between the curves $x = y^2$ and $x = 4$ is divided into two equal parts by the line $x = a$, then find the value of a, using integration.

\item Find the area bounded by the ellipse $x^2+4y^2=16$ and the ordinates $x=0$ and $x=2$, using integration.

\item Find the area of the region $\{(x,y) : x^2 \leq y \leq x\}$, using integration

\end{enumerate}

\section{Curves}
\input{curves.tex}
\chapter{Tangent And Normal}
\section{Equation of Tangents}
\begin{enumerate}

\item Find the equation of tangent to the curve $y = x^2 + 4x + 1$ at the point $(3,22)$.

\item The slope of the normal to the curve $y = 2x^2 +3 sin x$ at $x=0$ is \underline{\hspace{1cm}}.

\end{enumerate}


\section{Miscellaneous}
\input{misc.tex}

%\iffalse
%\chapter{Vectors}
%\section{Length}
%\input{chapters/vectors/examples/length.tex}
%\section{Distance}
%\input{chapters/vectors/examples/distance.tex}
%\section{Exercises}
%\input{chapters/vectors/exer/distance.tex}
%\section{Section Formula}
%\input{chapters/vectors/examples/section.tex}
%\section{Exercises}
%\input{chapters/vectors/exer/section.tex}
%\section{Rank}
%\input{chapters/vectors/examples/rank.tex}
%\section{Exercises}
%\input{chapters/vectors/exer/rank.tex}
%\section{Scalar Product}
%%\input{chapters/vectors/examples/scalar.tex}
%\section{Exercises}
%\input{chapters/vectors/exer/scalar.tex}
%\section{Orthogonality}
%\input{chapters/vectors/examples/ortho.tex}
%\section{Exercises}
%\input{chapters/vectors/exer/ortho.tex}
%\section{Vector Product}
%\input{chapters/vectors/examples/cross.tex}
%\section{Exercises}
%\input{chapters/vectors/exer/cross.tex}
%\section{Miscellaneous}
%\input{chapters/vectors/examples/misc.tex}
%\section{Exercises}
%\input{chapters/vectors/exer/misc.tex}
%\section{Triangle}
%\input{chapters/const/examples/tri.tex}
%\section{Exercises}
%\input{chapters/const/exer/tri.tex}
%\section{ Quadrilateral}
%\input{chapters/const/examples/quad.tex}
%\section{Exercises}
%\input{chapters/const/exer/quad.tex}
%
%\chapter{Linear Forms}
%\section{Equation of a Line}
%\input{chapters/linear/examples/equation.tex}
%\section{Perpendicular}
%\input{chapters/linear/examples/perp.tex}
%\section{Plane}
%\input{chapters/linear/examples/plane.tex}
%\section{Miscellaneous }
%\input{chapters/linear/examples/misc.tex}
%\section{Exemplar}
%\input{exemplar/11.10.3}
%\section{Singular Value Decomposition}
%\input{svd/svd.tex}
%
%\chapter{Constructions}
%\section{JEE}
%\input{jee/7.tex}
%--------------------------------------------------------
%\section{Properties}
%\input{chapters/9/9/3.tex}
%
%\section{Properties}
%\input{chapters/9/8/1.tex}
%\section{Mid Point Theorem}
%\input{chapters/9/8/2.tex}
%\section{Parallelograms}
%\section{Triangles and Parallelograms}
%\input{chapters/9/9/4.tex}
%--------------------------------------------------------

%\chapter{Circles}
%
%\chapter{Tangents to a Circle}


%\chapter{Conics}
%\section{Circle}
%\input{chapters/circles/examples/equation.tex}
%\section{Exercises}
%\input{chapters/circles/exer/equation.tex}
%\section{Construction}
%\input{chapters/circles/examples/const.tex}
%\section{Exercises}
%\input{chapters/circles/exer/const.tex}
%\section{Parabola}
%\input{chapters/conics/examples/parab.tex}
%section{Exercises}
%\input{chapters/conics/exer/parab.tex}
%\section{Ellipse}
%\input{chapters/conics/examples/ellipse.tex}
%\section{Exercises}
%\input{chapters/conics/exer/ellipse.tex}
%\section{Hyperbola}
%\input{chapters/conics/examples/hyper.tex}
%\section{Exercises}
%\input{chapters/conics/exer/hyper.tex}

%
%\fi


%\include{ch02} 
%\backmatter
%\appendix
%\iffalse
%\chapter{ Vectors}
%\section{$2\times 1$ vectors}
%\input{matrix/two.tex}
%\include{app01}
%\appendix
%\section{$3\times 1$ vectors}
%\input{matrix/three.tex}
%\chapter{Matrices}
%\input{matrix/mat.tex}
%\input{linman/chapters/decomp/svd.tex}

%\chapter{Triangle Constructions}
%\input{cons/tri.tex}


%\chapter{Linear Forms}
%\section{Two Dimensions}
%\input{linear/two.tex}
%\section{Three Dimensions}
%\input{linear/three.tex}
%\chapter{Quadratic Forms}
%\numberwithin{equation}{subsection}
%\numberwithin{equation}{section}
%\section{Conic equation }
%\input{quad/defs.tex}
%\section{Circles}
%\input{quad/circle.tex}

%\section{Standard Form}
%\input{quad/stddef.tex}
%\chapter{Conic Parameters}
%\section{Standard Form}
%\input{quad/standard.tex}
%\section{Quadratic Form }
%\input{quad/coroll.tex}

%\chapter{Conic Lines}
%\section{Pair of Straight Lines}
%
%\input{quad/pair.tex}
%\section{Intersection of Conics}
%\input{quadlines/inter.tex}
%\section{ Chords of a Conic}
%\input{quadlines/chord.tex}
%\section{ Tangent and Normal}
%\input{quadlines/tangent.tex}
%\fi
%\chapter{Proofs}
%   \section{}
%\input{apps/defs.tex}

%  \section{}
%\input{apps/parab.tex}
%  \section{}
%\input{apps/nonparab.tex}
%		\section{}
%\input{apps/params.tex}
%\latexprintindex


 
%\section{Examples}
%\subsection{Loney}
%\input{examples/loney.tex}
%\subsection{Miscellaneous}
%\input{examples/misc.tex}
%
%%\section*{Disclosure Statement}
%%The authors report there are no competing interests to declare.
%%
%%
%%
%%  
%%%All the results related to conics are summarized in 
%%%Table \ref{table:conics}.  
%%%\begin{table*}[!t]
%%%\centering
%%%\input{conics.tex}
%%%%\input{./figs/conics.tex}
%%%\caption{$\vec{x}^{\top}\vec{V}\vec{x}+2\vec{u}^{\top}\vec{x}+f = 0$  can be expressed in the above standard form for various conics. $\vec{c}$ represents the centre/vertex of the conic. $\vec{q}$ is/are the point(s) of contact for the tangent(s). }
%%%\label{table:conics}
%%%\end{table*}
%%%\begin{verbatim}
%%\bibliographystyle{tfs}
%%%\bibliography{interacttfssample}
%%\bibliography{school}
%%\end{verbatim}
%% included where the list of references is to appear, where \texttt{tfs.bst} is the name of the \textsc{Bib}\TeX\ bibliography style file for Taylor \& Francis' Reference Style S and \texttt{interacttfssample.bib} is the bibliographic database included with the \textsf{Interact}-TFS \LaTeX\ bundle (to be replaced with the name of your own .bib file). \LaTeX/\textsc{Bib}\TeX\ will extract from your .bib file only those references that are cited in your .tex file and list them in the References section.
%
%% Please include a copy of your .bib file and/or the final generated .bbl file among your source files if your .tex file does not contain a reference list in a \texttt{thebibliography} environment.
%

  % \section{Appendices}
  % \appendix
%			\appendices

\end{document}
